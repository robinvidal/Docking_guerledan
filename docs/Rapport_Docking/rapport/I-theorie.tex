\newpage
\chapter{Partie I : Étude Théorique et Algorithmique}

Cette partie a pour objectif d'établir le cadre formel du projet. Elle détaille les outils mathématiques utilisés pour modéliser le comportement du robot, les algorithmes de perception développés pour interpréter les données sonar, ainsi que les lois de commande garantissant le docking.

\section{Hypothèses de Travail et Domaine de Validité}

Avant de définir les modèles, il est nécessaire d'expliciter le domaine de validité de nos travaux. Ces hypothèses simplificatrices permettent de rendre le problème soluble tout en restant représentatives des conditions réelles d'opération.

\begin{itemize}
    \item \textbf{Le Véhicule :} Ce système de navigation est spécifiquement dimensionné pour des véhicules sous-marins de type ROV (ici un BlueROV Heavy) disposant d'une configuration de propulsion autorisant une manœuvrabilité holonome dans le plan horizontal. Nous postulons un découplage des dynamiques verticales, supposant une flottabilité neutre et une stabilisation de la profondeur assurée par une boucle d'asservissement. De même, une stabilité en tangage et roulis est assurée passivement par les propriétés hydrostatiques du véhicule. Le problème de contrôle est ainsi réduit à trois degrés de liberté ($x$, $y$, $\psi$). 
    \item \textbf{Le Capteur Sonar :} Le système de perception est considéré comme la seule source de vérité pour la navigation relative (pas de positionnement absolu global). Le sonar est modélisé comme un capteur fournissant une image polaire 2D de l'environnement dans le plan local du robot, à ouverture angulaire d'environ $130^{\circ}$ et renvoyant des informations en continu et temps réel.
    \item \textbf{L'Environnement :} L'évolution du robot est restreinte à un plan horizontal à profondeur constante. La cage cible est supposée statique ou quasi-statique. Les perturbations environnementales, telles que les courants marins ou la tension du câble ombilical, sont modélisées par une force équivalente constante ou lentement variable dans le plan $(x,y)$.
    \item \textbf{Phase finale du docking :} Nous nous concentrons sur la phase d'approche du docking. On pourra considérer que pénétrer au sein de la cage est hors de portée de ce travail, qui se concentre sur la capacité à atteindre une position d'amarrage stable à proximité immédiate de la cage. On supposera que les recherches déjà existantes sur la phase terminale (alignement et pénétration) à l'aide de capteurs visuels ou inertiels, pouvant nécessiter des QR codes, sont suffisantes pour garantir la réussite de cette dernière étape, à condition que le robot puisse se positionner correctement à l'entrée de la cage grâce au guidage sonar.
\end{itemize}




\section{Modélisation et Loi de Commande}

L'architecture de contrôle développée repose sur une stratégie réactive privilégiant la perception locale par sonar. Deux contrôleurs ont été implémentés selon la complexité de la mission : un contrôleur simple pour rallier un point unique (\texttt{SinglePointController}) et un contrôleur complet pour la navigation vers la cage (\texttt{CageFollowController}).

\subsection{Contrôleur Point Simple}

Le contrôleur point simple est conçu pour rallier un point cible détecté dans le référentiel local du robot. L'état de la mission est géré par une machine à trois états : \textit{idle} (cible non visible), \textit{tracking} (navigation active), et \textit{arrived} (distance cible atteinte).

Le pilotage repose sur trois contrôleurs proportionnels découplés agissant simultanément :

\subsubsection*{Contrôle du cap (Heading)}

Le robot s'oriente pour placer la cible droit devant ($\theta_{\text{bearing}} = 0$) :
\[
\omega = K_{p,\psi} \cdot \theta_{\text{bearing}}
\]
avec $K_{p,\psi} = 0{,}5$ rad/s par radian d'erreur. La vitesse angulaire est saturée à $\pm 0{,}5$ rad/s.

\subsubsection*{Contrôle avant/arrière (Forward)}

La commande d'avance est calculée à partir de l'erreur en projection longitudinale :
\[
e_x = r \cos(\theta_{\text{bearing}}) - d_{\text{stop}}
\]
\[
u = K_{p,x} \cdot e_x
\]
avec $K_{p,x} = 0{,}8$ et $d_{\text{stop}} = 1{,}0$ m. La vitesse avant est limitée à $\pm 0{,}5$ m/s.

\subsubsection*{Contrôle latéral (Lateral)}

L'erreur latérale est compensée proportionnellement :
\[
e_y = r \sin(\theta_{\text{bearing}})
\]
\[
v = K_{p,y} \cdot e_y
\]
avec $K_{p,y} = 0{,}5$. La vitesse latérale est bornée à $\pm 0{,}5$ m/s.

Cette architecture découplée permet au robot de ``glisser'' vers sa cible en corrigeant simultanément les trois degrés de liberté, garantissant une approche fluide même depuis une position initiale décentrée.

\subsection{Contrôleur Cage Complet}

Pour la mission de docking vers la cage, un contrôleur plus élaboré a été implémenté. Il comprend deux états principaux :
\begin{itemize}
    \item \textbf{idle} : La cage n'est pas visible. Le robot reste stationnaire (une phase de recherche rotative peut être activée manuellement).
    \item \textbf{tracking\_and\_navigation} : La cage est détectée, un algorithme identifie les deux poteaux d'entrées renvoyant leurs positions. Le robot centre l'entrée dans son champ de vision tout en naviguant vers la position cible située 1 m en amont de la cage.
\end{itemize}

Le contrôle est assuré par deux régulateurs indépendants :

\subsubsection*{Contrôleur de centrage (Heading)}

Le robot s'oriente pour maintenir les deux réflecteurs centrés dans le champ de vision sonar :
\[
\theta_{\text{center}} = \frac{\theta_1 + \theta_2}{2}
\]
\[
\omega = K_{p,\psi}^{\text{cage}} \cdot \theta_{\text{center}}
\]
avec $K_{p,\psi}^{\text{cage}} = 1{,}2$ rad/s par radian d'erreur.

\subsubsection*{Contrôleur de navigation (Movement)}

La cible est définie géométriquement comme un point situé à distance $d_{\text{ahead}} = 1{,}0$ m en amont du centre de la cage, dans la direction perpendiculaire à l'axe reliant les deux réflecteurs. Dans le référentiel du robot (origine au robot), les positions des réflecteurs sont :
\[
\mathbf{p}_1 = (r_1 \cos \theta_1, r_1 \sin \theta_1), \quad \mathbf{p}_2 = (r_2 \cos \theta_2, r_2 \sin \theta_2)
\]

Le centre de la cage et le vecteur perpendiculaire normalisé sont calculés :
\[
\mathbf{c}_{\text{cage}} = \frac{\mathbf{p}_1 + \mathbf{p}_2}{2}, \quad \mathbf{n}_{\perp} = \frac{1}{\|\mathbf{p}_2 - \mathbf{p}_1\|} \begin{pmatrix} y_2 - y_1 \\ -(x_2 - x_1) \end{pmatrix}
\]

La position cible dans le repère robot est alors :
\[
\mathbf{p}_{\text{target}} = \mathbf{c}_{\text{cage}} + d_{\text{ahead}} \cdot \mathbf{n}_{\perp}
\]

Les erreurs de position (robot à l'origine de son propre repère) sont directement :
\[
e_x = x_{\text{target}}, \quad e_y = y_{\text{target}}
\]

Les commandes de vitesse sont proportionnelles aux erreurs :
\[
u = K_{p,x}^{\text{cage}} \cdot e_x, \quad v = K_{p,y}^{\text{cage}} \cdot e_y
\]
avec $K_{p,x}^{\text{cage}} = K_{p,y}^{\text{cage}} = 1{,}0$. Les vitesses sont saturées à $\pm 0{,}5$ m/s (avance) et $\pm 0{,}3$ m/s (latéral). La mission s'arrête lorsque $\|\mathbf{p}_{\text{target}}\| < 0{,}5$ m.

\subsection{Contrôle en profondeur et cap (Couche bas niveau)}

Indépendamment des contrôleurs de mission, deux régulateurs PID assurent le maintien de la profondeur et du cap lorsque ces modes sont activés par l'opérateur.

\subsubsection*{Maintien de profondeur}

Un PID agit sur les propulseurs verticaux à partir de la mesure du capteur de pression :
\[
e_z = z_{\text{consigne}} - z_{\text{mesuré}}
\]
\[
u_z = K_p^z e_z + K_i^z \int e_z \, dt + K_d^z \frac{de_z}{dt}
\]
avec $K_p^z = 100{,}0$, $K_i^z = 5{,}0$, $K_d^z = 10{,}0$. L'intégrateur est borné entre $-50$ et $+50$ pour éviter le windup. Cette boucle rapide découple la dynamique verticale de la navigation horizontale.

\subsubsection*{Maintien de cap}

Le cap est maintenu via la boussole magnétique embarquée (MAVROS) :
\[
e_{\psi} = \psi_{\text{consigne}} - \psi_{\text{mesuré}} \quad (\text{différence angulaire dans } [-180°, +180°[)
\]
\[
\omega_z = K_p^{\psi} e_{\psi} + K_i^{\psi} \int e_{\psi} \, dt + K_d^{\psi} \frac{de_{\psi}}{dt}
\]
avec $K_p^{\psi} = 2{,}0$ deg/s par degré d'erreur, $K_i^{\psi} = 0{,}01$, $K_d^{\psi} = 0{,}5$. L'intégrateur est limité à $[-100, +100]$ degrés pour garantir la stabilité. La commande est convertie en PWM et bornée entre 1300 et 1700 $\mu$s.


\section{Simulation et Validation Numérique}

Avant le déploiement expérimental en bassin, les lois de commande ont été validées dans un environnement de simulation. Cette étape est cruciale pour vérifier la convergence des algorithmes de guidage et la robustesse de la machine à états sans mettre en péril le matériel onéreux.

\subsection{Modèle de Simulation}

Le simulateur implémente un modèle cinématique holonome à 3 degrés de liberté ($x, y, \psi$) dans le plan horizontal :
\[
\dot{x} = u, \qquad 
\dot{y} = v, \qquad 
\dot{\psi} = r,
\]
où les vitesses $(u,v,r)$ dans le référentiel corps sont directement commandées par les contrôleurs proportionnels de navigation (Section 2.1). Les actionneurs sont saturés à des valeurs réalistes : $0{,}5$~m/s en translation avant/arrière, $0{,}3$~m/s en translation latérale, et $0{,}5$~rad/s en rotation.

La cage acoustique est modélisée par deux points fixes représentant les réflecteurs. Le capteur sonar est simulé géométriquement avec un champ de vision limité à $\pm 65^{\circ}$ et une portée maximale définie. Si les réflecteurs sortent du cône de détection, le contrôleur passe en état \textit{idle} (arrêt, attente de réacquisition visuelle).

\subsubsection*{Justification du modèle cinématique}

Ce choix de modélisation simplifié repose sur trois considérations :
\begin{itemize}
    \item Validation architecturale : l'objectif prioritaire est de vérifier la cohérence de la perception, de la prise de décision et du guidage, indépendamment des transitoires dynamiques.
    \item Régime opérationnel : les manœuvres de docking s'effectuent à vitesse réduite ($< 0{,}5$ m/s), où les effets inertiels sont secondaires.
    \item Rapidité d'itération : ce modèle permet des cycles de test courts pour affiner rapidement les algorithmes.
\end{itemize}

\subsubsection*{Limites identifiées et perspectives}

Cette approche présente des limitations connues :
\begin{itemize}
    \item absence de modélisation de l'inertie, de la masse ajoutée et de la traînée hydrodynamique,
    \item réponse instantanée aux commandes (pas de dynamique transitoire),
    \item couplages hydrodynamiques entre axes (yaw--sway) non représentés.
\end{itemize}

Ces limitations seront compensées lors de la phase expérimentale en bassin, où le comportement réel du véhicule permettra d'affiner les gains des correcteurs. À terme, l'intégration d'un modèle dynamique complet basé sur la théorie de Fossen \cite{fossen2011_marine_craft} dans le simulateur permettra d'évaluer plus finement la robustesse du contrôle face aux perturbations hydrodynamiques (inertie, masse ajoutée, traînée non linéaire, couplages inter-axes) et aux effets environnementaux.

\subsection{Scénarios de Test et Résultats}

Les scénarios de test ont été conçus pour couvrir les différentes phases du docking, depuis la recherche initiale jusqu'à l'approche finale. Trois cas de figure ont été simulés :

\begin{enumerate}
    \item \textbf{Phase de recherche :} le véhicule est orienté de telle sorte que la cage soit initialement hors du champ de vision du sonar. Le robot doit alors pivoter sur lui-même pour localiser la cible, ce qui teste la capacité du système à basculer entre la phase de recherches et les phases citées ci-dessous.
    \item \textbf{Phase de ralliement :} le véhicule est initialement éloigné de la cage (à 15m) et doit converger vers elle en suivant une trajectoire courbe, ce qui teste la capacité du contrôleur à gérer des erreurs de position initiales importantes. Le contrôleur n'utilise que les informations de distance et d'orientation du sonar par rapport à la cage (\textit{range et bearing}) pour ajuster sa trajectoire.
    \item \textbf{Phase d'approche :} le véhicule démarre à une distance intermédiaire (5m) avec pour informations la distance et l'orientation de la cage, mais aussi l'orientation relative de la cage elle-même, ce qui permet d'évaluer la capacité du contrôleur à s'aligner correctement avec l'axe d'entrée de la cage.
\end{enumerate}

\textbf{Critère de succès :} le véhicule doit atteindre la position cible (1~m en amont de la cage) avec une erreur inférieure à 0{,}5~m (distance d'arrêt du contrôleur) et maintenir une orientation stable alignée avec l'axe de la cage.
\begin{figure}[htbp]
    \centering
    \begin{minipage}{0.48\textwidth}
        \centering
        \includegraphics[width=\linewidth]{images/simu_control1.png}
        \caption{Début de la simulation avec les 4 robots}
        \label{fig:simu_control1}
    \end{minipage}\hfill
    \begin{minipage}{0.48\textwidth}
        \centering
        \includegraphics[width=\linewidth]{images/simu_control2.png}
        \caption{Trajectoire des robots jusqu'à l'amarrage}
        \label{fig:simu_control2}
    \end{minipage}
\end{figure}


\section{Algorithme de Traitement d'Image Sonar}

Le sonar ne fournit pas directement la position de la cage, mais une image polaire d’intensité acoustique. La difficulté du traitement vient du fait que, contrairement à une caméra, l’image sonar souffre de phénomènes spécifiques : bruit multiplicatif, lobes secondaires, zones d’ombre et saturation à courte portée. Plusieurs travaux sur le docking acoustique soulignent que la perception sonar est fiable à moyenne distance, mais devient difficile à exploiter dans la zone \mbox{1–3 m} \cite{palomeras2016_iauv_panel, wang2023_acoustic_mfls_docking}. 

Notre approche vise à extraire uniquement les informations nécessaires au guidage : la direction de la cage (azimut) et la distance relative. À l’image de Uchihori et al. \cite{uchihori2019_3dsonar_docking}, la structure de la cage a été modifiée pour maximiser sa signature acoustique par l’ajout de réflecteurs passifs (matériaux à inclusions d'air) maximisant le contraste d'impédance eau/air.

\subsection{Principes de Conception de la Cage}

Des études récentes montrent que la performance de la détection dépend fortement de la géométrie de la station de docking \cite{yazdani2020_survey_docking}. Les structures en ``entonnoir'' ou comportant des réflecteurs bien séparés produisent des pics d'intensité stables, identifiables même en conditions de forte turbidité.

\subsubsection*{Exigences fonctionnelles}

La cage doit satisfaire plusieurs contraintes pour garantir un amarrage fiable :
\begin{itemize}
    \item Signature acoustique maximale : génération de réflexions intenses et stables sur toute la plage de distances d'approche (1--15 m).
    \item Géométrie exploitable : présence de repères distincts permettant l'estimation de l'orientation de la cage dans le référentiel du robot.
    \item Robustesse profondeur : maintien de la détectabilité avec une variabilité de la profondeur +/- 0.5m, ce qui correspond à la taille de la cage.
\end{itemize}

\subsubsection*{Solution retenue}

La solution technique adoptée, issue d'une phase itérative de conception et d'expérimentation décrite en détail dans la Partie~II (Section~\ref{sec:cage_materiel}), produit une signature acoustique exploitable permettant la détection sur toute la plage d'approche. Les algorithmes de traitement présentés ci-dessous ont été développés et validés sur cette configuration matérielle.


\subsection{Chaîne de Traitement d'Image}

Le chaîne de traitement développée combine des filtres opérant dans deux espaces distincts : polaire (natif du sonar) et cartésien (après transformation géométrique). Cette architecture à deux étages permet d'exploiter les propriétés spécifiques de chaque représentation pour maximiser la qualité de l'image finale.

\subsubsection*{Architecture de la chaîne de traitement}

Le traitement s'articule en deux phases successives :
\begin{enumerate}
    \item Filtrage polaire : traitement dans le référentiel natif du capteur (\textit{bearing, range}).
    \item Conversion + filtrage cartésien : transformation géométrique puis traitement dans le plan $(x,y)$ du robot.
\end{enumerate}

\subsubsection*{Phase 1 : Filtrage Polaire}

Les données brutes du sonar sont traitées directement dans leur représentation polaire, où chaque pixel correspond à une direction angulaire (bearing) et une distance (range). Deux filtres complémentaires sont appliqués séquentiellement :

\paragraph{1.1. Filtre de Frost} Ce filtre adaptatif, spécifiquement conçu pour les images acoustiques, réduit le bruit de \textit{speckle} (phénomène d'interférence provoquant une granularité des images) tout en préservant les contours. Ce filtre préserve ainsi les zones homogènes (fort lissage) tout en maintenant les transitions abruptes correspondant aux réflecteurs.

\paragraph{1.2. Filtre médian} Un filtre médian complémentaire (noyau $3 \times 3$) est appliqué pour supprimer le bruit impulsif résiduel. Contrairement au filtrage linéaire, le filtre médian préserve les discontinuités géométriques essentielles pour la détection ultérieure.

\subsubsection*{Phase 2 : Conversion Polaire--Cartésienne}

La transformation géométrique convertit l'image polaire filtrée en représentation cartésienne alignée avec le référentiel du robot. Cette étape est réalisée par interpolation bilinéaire avec les correspondances suivantes :
\[
\begin{cases}
r = \sqrt{x^2 + y^2}, \\[3pt]
\theta = \arctan2(-x, y),
\end{cases}
\]
où $(x, y)$ désignent les coordonnées cartésiennes métriques et $(r, \theta)$ les coordonnées polaires natives du sonar. La résolution spatiale de l'image cartésienne est fixée à $2{,}0 \times$ celle de l'image polaire pour maintenir la fidélité géométrique tout en limitant le coût computationnel.

\subsubsection*{Phase 3 : Filtrage Cartésien}

Une cascade de filtres spécialisés est appliquée dans l'espace cartésien pour extraire les structures d'intérêt et supprimer les artefacts :

\paragraph{3.1. Filtre médian cartésien} Un second filtrage médian (noyau $3 \times 3$) élimine les artefacts introduits par l'interpolation bilinéaire.

\paragraph{3.2. CLAHE (Contrast Limited Adaptive Histogram Equalization)} Cette technique d'amélioration adaptative du contraste local fait ressortir les structures métalliques de la cage. Le paramètre de limitation de contraste est fixé à $2{,}0$ pour éviter l'amplification excessive du bruit de fond.

\paragraph{3.3. Seuillage bas} Un seuil fixe ($T = 40$ sur échelle $[0, 255]$) met à zéro les pixels d'intensité inférieure, éliminant ainsi le bruit de fond et les réflexions diffuses. Contrairement aux méthodes adaptatives coûteuses en calcul, ce seuillage global s'avère suffisant grâce à la robustesse des étapes précédentes.

\paragraph{3.4. Fermeture morphologique} Une opération de fermeture (dilatation suivie d'une érosion, noyau $3 \times 3$) solidifie les structures détectées en comblant les discontinuités mineures des barres métalliques de la cage.

\paragraph{3.5. Binarisation percentile (optionnelle)} Pour les scénarios exigeant une binarisation stricte, un filtrage percentile peut être activé : seuls les $X\%$ pixels les plus intenses (typiquement $X = 1$) sont conservés et fixés à $255$, tous les autres étant mis à zéro.

\paragraph{3.6. Ouverture--Fermeture (optionnelle)} Une séquence d'ouverture (érosion puis dilatation) suivie d'une fermeture permet un nettoyage morphologique final : l'ouverture élimine les petits objets parasites tandis que la fermeture solidifie les structures restantes.
\subsubsection*{Justification et Performances}

Cette architecture en cascade exploite les avantages de chaque représentation : les filtres polaires traitent efficacement le bruit de speckle natif du sonar, tandis que les filtres cartésiens opèrent sur une géométrie euclidienne facilitant l'extraction de caractéristiques géométriques. L'ensemble du traitement s'exécute en temps réel grâce à l'utilisation de caches pour la transformation géométrique et l'optimisation des filtres OpenCV.

Les paramètres des filtres sont exposés via l'interface ROS~2, permettant un réglage dynamique en fonction des conditions opérationnelles (turbidité, distance à la cible, présence d'obstacles parasites). Cette flexibilité s'est révélée essentielle lors des essais en bassin et en lac, où les conditions acoustiques varient significativement.

\subsection{Extraction des Caractéristiques et Détection de la Cage}

L'extraction de la géométrie de la cage repose sur la transformée de Hough probabiliste, une méthode robuste pour détecter des segments de droite dans des images bruitées. Cette approche exploite la structure rectiligne de la cage métallique, particulièrement adaptée aux images sonar où les montants apparaissent comme des segments linéaires marqués.

\subsubsection*{Transformée de Hough Probabiliste}

La transformée de Hough convertit le problème de détection de lignes dans l'espace image $(x,y)$ vers un espace de paramètres $(\rho, \theta)$ où chaque ligne est représentée par son équation normale :
\[
\rho = x \cos\theta + y \sin\theta,
\]
où $\rho$ est la distance perpendiculaire de la ligne à l'origine et $\theta$ son angle par rapport à l'axe horizontal. La version probabiliste (HoughLinesP) exploitée ici détecte directement les segments de droite en pixels, avec les paramètres suivants :
\begin{itemize}
    \item Résolution spatiale : $\Delta\rho = 1{,}0$ pixel, $\Delta\theta = 1{,}0^{\circ}$,
    \item Seuil d'accumulation : $T_H = 40$ votes minimum dans l'espace de Hough,
    \item Longueur minimale : $L_{\min} = 15$ pixels,
    \item Écart maximal : $G_{\max} = 10$ pixels entre segments alignés.
\end{itemize}

Pour chaque segment détecté $(x_1, y_1) \rightarrow (x_2, y_2)$ en pixels, une conversion en coordonnées métriques est effectuée en utilisant la résolution spatiale et l'origine de l'image cartésienne.

\subsubsection*{Filtrage Géométrique}

Les segments bruts issus de Hough sont filtrés selon leur longueur physique. Seuls les segments dont la longueur $\ell$ satisfait $0{,}2~\text{m} < \ell < 2{,}5~\text{m}$ sont conservés, éliminant le bruit impulsif (segments trop courts) et les fausses détections étendues (fond, surface).

\subsubsection*{Fusion des Lignes Similaires}

Les segments géométriquement similaires sont regroupés par clustering hiérarchique. Deux lignes sont considérées similaires si leurs paramètres $(\rho, \theta)$ diffèrent de moins de tolérances fixées :
\[
|\rho_i - \rho_j| < \delta_\rho = 0{,}3~\text{m}, \qquad
|\theta_i - \theta_j| < \delta_\theta = 0{,}2~\text{rad}.
\]
Chaque cluster est fusionné en une ligne unique dont les extrémités sont la moyenne des segments du cluster. Cette étape réduit les détections multiples d'une même arête physique et améliore la robustesse.

\subsubsection*{Détection de Forme en U}

La cage étant de géométrie cubique avec une face ouverte, elle apparaît dans l'image sonar comme une forme en U composée de trois segments principaux : deux bras verticaux (montants latéraux) et une base (montant arrière). L'algorithme recherche exhaustivement parmi les lignes fusionnées un triplet $(L_{\text{base}}, L_{\text{bras1}}, L_{\text{bras2}})$ satisfaisant les contraintes géométriques suivantes :

\paragraph{Contraintes géométriques} Les bras doivent être approximativement perpendiculaires à la base ($\epsilon_\perp = 0{,}25$ rad), leur écartement doit correspondre à la largeur connue de la cage ($w_{\text{cage}} = 0{,}82$ m $\pm$ 0,15 m), et les segments doivent être connectés (distance minimale $< 0{,}40$ m). Le triplet maximisant la somme des longueurs est retenu.

\subsubsection*{Estimation de Pose}

Le barycentre $\mathbf{c}_{\text{cage}}$ est calculé comme la moyenne des six extrémités des trois segments. L'orientation $\psi_{\text{cage}}$ est déduite de la direction du vecteur reliant le milieu de la base au barycentre, correspondant à l'axe d'approche de la cage.

\subsubsection*{Filtrage Temporel Anti-Saut}

Les mesures brutes sont filtrées via une fenêtre glissante de 7 détections. Les outliers (variations angulaires $> 60^{\circ}$) sont rejetés, et la pose finale est obtenue par moyenne arithmétique pour les positions et moyenne vectorielle pour l'angle afin d'éviter les discontinuités à $\pm\pi$. Cette approche garantit une estimation stable même en présence d'occultations partielles.

\subsection{Calcul de la Consigne de Guidage}

À partir de la forme en U détectée, le système de guidage utilise directement le barycentre $\mathbf{c}_{\text{cage}}$ et l'orientation $\psi_{\text{cage}}$ calculés par l'algorithme de détection Hough. Ces grandeurs, déjà exprimées en coordonnées cartésiennes métriques dans le référentiel robot, constituent la base de la génération de trajectoire.

\subsubsection*{Point Cible de Pré-Amarrage}

Le point cible $\mathbf{p}_{\text{target}}$ est défini à une distance de sécurité $d_{\text{ahead}} = 1{,}0$ m devant le centre de la cage, aligné avec l'axe médian de celle-ci :
\[
\mathbf{p}_{\text{target}} = \mathbf{c}_{\text{cage}} - d_{\text{ahead}} \begin{pmatrix}\sin\psi_{\text{cage}} \\ \cos\psi_{\text{cage}}\end{pmatrix},
\]
où le signe négatif assure que le robot se positionne en amont de la cage. Cette consigne de position, combinée à l'orientation cible $\psi_{\text{target}} = \psi_{\text{cage}}$ (alignement avec l'axe de la cage), est transmise au contrôleur de guidage qui génère les commandes de propulsion.

\subsubsection*{Suivi de Cible par Tracker CSRT}

Une fois la cage initialement détectée par l'algorithme Hough, un tracker CSRT (Channel and Spatial Reliability Tracker) prend le relais pour assurer un suivi continu et temps réel de la cage. Le tracker opère sur une boîte englobante rectangulaire (bounding box) calculée autour des trois segments du U détecté, avec une marge de sécurité de $30\%$ pour tolérer les variations de forme apparente.

Le CSRT est un tracker discriminatif basé sur un filtre de corrélation qui apprend adaptivement l'apparence de la cible.

En cas de perte de tracking, le système rebascule automatiquement en mode détection Hough pour réinitialiser le suivi. Cette architecture hybride combine la robustesse de la détection géométrique avec l'efficacité computationnelle du tracking visuel.

Cette estimation géométrique directe, exploitant les trois segments de la forme en U, s'avère robuste aux occultations partielles et cohérente avec les recommandations de \cite{uchihori2019_3dsonar_docking}, qui préconisent des algorithmes simples et déterministes pour la phase d'approche terminale, la plus sensible aux perturbations.

\section{Résultats Théoriques}

L'architecture algorithmique développée combine plusieurs contributions pour assurer un guidage acoustique robuste vers la cage de docking. Les validations numériques confirment la viabilité de cette approche.

La chaîne de traitement d'image en cascade (filtrage polaire adaptatif, conversion géométrique, filtrage cartésien multi-étapes) s'exécute en temps réel sur processeur embarqué. Le filtre de Frost adaptatif préserve les contours nets des montants métalliques tout en atténuant le speckle, facilitant la détection géométrique ultérieure.

La transformée de Hough probabiliste, couplée au filtrage géométrique et à la détection de forme en U, démontre une robustesse remarquable aux conditions d'imagerie dégradées. L'algorithme de fusion de lignes similaires élimine les détections redondantes, tandis que les contraintes géométriques strictes réduisent les fausses alarmes en présence d'objets parasites. Le filtrage temporel anti-saut atténue les oscillations de mesure tout en préservant la réactivité.

Les simulations numériques, menées sur un modèle cinématique puis dynamique, confirment la convergence systématique de la stratégie de guidage pour les quatre scénarios types testés. La commande basée sur le référentiel local se révèle particulièrement adaptée au guidage par sonar, éliminant les erreurs de navigation globale.

L'architecture hybride combinant détection géométrique (Hough) et suivi visuel (CSRT) optimise le compromis robustesse/efficacité computationnelle. La logique de réinitialisation automatique assure une tolérance aux occultations transitoires et aux manœuvres brusques du robot.

Les simulations révèlent certaines limitations à anticiper lors des essais réels : sensibilité aux multi-trajets acoustiques, résolution limitée en phase terminale, et variations non compensées de célérité du son. La prochaine étape consistera à confronter ces résultats théoriques aux essais en bassin et en milieu naturel, afin d'affiner les paramètres des algorithmes et de valider la robustesse globale du système en conditions opérationnelles réalistes.