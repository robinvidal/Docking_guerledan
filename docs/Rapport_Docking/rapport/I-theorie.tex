\newpage
\chapter{Partie I : Étude Théorique et Algorithmique}

Cette partie a pour objectif d'établir le cadre formel du projet. Elle détaille les outils mathématiques utilisés pour modéliser le comportement du robot, les algorithmes de perception développés pour interpréter les données sonar, ainsi que les lois de commande garantissant l'amarrage.

\section{Hypothèses de Travail et Domaine d'Emploi}

Avant de définir les modèles, il est nécessaire de circonscrire le domaine de validité de nos travaux. Ces hypothèses simplificatrices permettent de rendre le problème traitable tout en restant représentatives des conditions réelles d'opération.

\begin{itemize}
    \item \textbf{Le Véhicule :} Ce système de navigation est spécifiquement dimensionné pour des véhicules sous-marins de type ROV (ici un BlueROV Heavy) disposant d'une configuration de propulsion vectorielle, autorisant une manœuvrabilité holonome dans le plan horizontal. Nous postulons un découplage des dynamiques verticales, supposant une stabilisation de la profondeur et de l'assiette (tangage et roulis) assurée passivement par la conception hydrostatique ou par une boucle de régulation distincte. Le problème de contrôle est ainsi réduit à trois degrés de liberté ($x$, $y$, $\psi$).
    \textit{[Instruction manquante : Confirmer ici la matrice d'inertie approximative si connue, ou préciser que le modèle de traînée est supposé quadratique.]}

    \item \textbf{Le Capteur Sonar :} Le système de perception est considéré comme la seule source de vérité pour la navigation relative (pas de positionnement absolu global). Le sonar est modélisé comme un capteur fournissant une projection 2D de l'environnement dans le plan local du robot, extrayant l'azimut et la distance des amers (bouées).
    \textit{[Instruction manquante : Préciser ici les spécifications du sonar Oculus (ex : ouverture angulaire 130°, fréquence de rafraîchissement en Hz) et le modèle de bruit retenu pour la simulation (ex : bruit blanc gaussien sur la mesure de distance).]}

    \item \textbf{L'Environnement :} L'évolution du robot est restreinte à un plan horizontal à profondeur constante. La "cage" cible est supposée statique ou quasi-statique. Les perturbations environnementales, telles que les courants marins ou la tension du câble ombilical, sont considérées comme des forces constantes ou lentement variables que le contrôleur devra rejeter.
    \textit{[Instruction manquante : Quantifier l'intensité maximale du courant simulé (ex : < 0.5 m/s) et préciser si des obstacles autres que les bouées sont pris en compte.]}
\end{itemize}

\section{Modélisation et Loi de Commande}

Pour asservir la position du robot par rapport à la cage, nous avons synthétisé un contrôleur adapté aux contraintes de la perception locale.

La stratégie de contrôle repose sur une architecture réactive à deux états qui privilégie exclusivement la perception locale du robot, sans recours au GPS. Initialement, si le champ de vision ne capture pas les deux bouées, le robot active un mode de recherche en pivotant sur lui-même jusqu'à l'acquisition des cibles. Une fois le contact visuel établi, l'algorithme reconstruit géométriquement la "porte" formée par les bouées directement dans le référentiel du robot (où ce dernier est toujours à l'origine $[0,0]$) pour calculer un point de consigne virtuel situé un mètre devant cette porte.

Le pilotage est alors découplé en trois axes indépendants (avance, dérive latérale et cap) régulés par des boucles PID complètes. Dans cette configuration, l'action \textbf{Proportionnelle} transforme directement les coordonnées locales de la cible ($x$ et $y$) en commandes de vitesse pour assurer une réactivité immédiate, tandis que le terme \textbf{Intégral} s'avère indispensable pour compenser les perturbations constantes, telles que les courants marins ou la traînée du câble, qui créeraient sinon une erreur stationnaire. Enfin, la composante \textbf{Dérivée} amortit la réponse du système 

[Image of PID control loop block diagram]
 pour stabiliser le robot face à la forte inertie et aux effets hydrodynamiques. Cette méthode permet au robot de "glisser" vers sa destination en corrigeant les erreurs de position relative en temps réel, garantissant une approche fluide et précise même si le robot est initialement décentré.
\subsection{Contrôle en profondeur}
\textit{[Instruction : Expliquez brièvement comment la profondeur est maintenue constante (ex : PID sur la profondeur avec un capteur de pression).]}

\subsection{Modèle Dynamique du ROV}
\textit{[Instruction : Présentez ici les équations du mouvement de Fossen sous forme matricielle $M\dot{\nu} + C(\nu)\nu + D(\nu)\nu + g(\eta) = \tau$. Détaillez brièvement ce que contient chaque matrice (Masse ajoutée, Coriolis, Amortissement quadratique) pour le BlueROV.]}

\subsection{Synthèse de la Loi de Commande}
\textit{[Instruction : C'est le cœur de la commande. Présentez la structure de votre contrôleur (PID vectoriel, LQR, Commande Prédictive MPC, ou Mode Glissant). Écrivez l'équation de la commande $\tau = ...$ et expliquez le rôle de chaque terme (terme proportionnel pour l'erreur, intégral pour le courant, etc.). Justifiez ce choix par rapport aux capacités de calcul du robot.]}

\section{Simulation et Validation Numérique}

Avant le déploiement expérimental en bassin, la loi de commande a été validée dans un environnement virtuel. Cette étape est cruciale pour vérifier la convergence des algorithmes de guidage et la robustesse de la machine à états (basculement entre recherche et suivi) sans mettre en péril le matériel onéreux.

\subsection{Environnement de Simulation}

Pour le prototypage rapide de la loi de commande, nous avons développé un simulateur cinématique simple sous le langage Python. Contrairement aux simulateurs dynamiques lourds, cet outil permet de valider la logique décisionnelle en temps réel accéléré.
Le modèle du robot implémenté est un modèle holonome à 3 degrés de liberté ($x, y, \psi$), où les commandes de vitesse $(u, v, r)$ sont intégrées directement pour obtenir la position, avec une saturation réaliste des actionneurs (fixée à $0.5$ m/s et $0.5$ rad/s dans nos tests).

La "cage" acoustique est modélisée par deux points singuliers fixes dans l'espace, représentant les échos des deux bouées verticales. Le capteur (sonar ou caméra) est simulé de manière géométrique avec une contrainte majeure : un champ de vision (FOV) limité à $\pm 45^{\circ}$. Si une bouée sort de ce cône ou dépasse la portée maximale, l'information de distance et de cap est perdue, forçant le contrôleur à réagir.

\subsection{Scénarios de Test et Résultats}
Pour éprouver la robustesse du contrôleur PID local, nous avons défini un scénario multi-robots impliquant quatre configurations initiales distinctes, visant à tester trois classes de manœuvres critiques :
\begin{enumerate}
    \item \textbf{Cas Nominal (Approche Directe et Correction Latérale) :} Ce scénario valide l'approche initiale et la correction latérale simple en positionnant le véhicule face à la cible mais avec un décalage par rapport à l'axe central.
    \item \textbf{Cas Angulaire (Alignement Couplé) :} Les véhicules sont placés loin sur les flancs (à $\pm 2.5$m). Ce scénario exige un couplage fort $Yaw/Sway$ pour que le contrôleur puisse aligner le cap tout en convergeant simultanément vers l'axe central de la porte.
    \item \textbf{Cas "Aveugle" (Déclenchement de la Recherche) :} Le véhicule est orienté de telle sorte que les bouées sont initialement hors de son champ de vision. Ce cas teste le déclenchement immédiat de l'état "Search", la rotation autonome pour l'acquisition de la cible, puis le verrouillage (Tracking).
\end{enumerate}

\textbf{Critère de succès :} La simulation est considérée comme validée lorsque le robot stabilise sa position à moins de $10$ cm du point de consigne virtuel (situé $1$ m devant le plan des bouées) et que les vitesses résiduelles sont nulles.
Les résultats numériques montrent une convergence systématique pour les quatre cas de figure en moins de 120 secondes, confirmant que la stratégie basée sur le référentiel local est suffisante pour accomplir la mission sans positionnement absolu.


\begin{figure}[htbp]
    \centering
    \begin{minipage}{0.48\textwidth}
        \centering
        \includegraphics[width=\linewidth]{images/simu_control1.png}
        \caption{Début de la simulation avec les 4 robots}
        \label{fig:simu_control1}
    \end{minipage}\hfill
    \begin{minipage}{0.48\textwidth}
        \centering
        \includegraphics[width=\linewidth]{images/simu_control2.png}
        \caption{Trajectoire des robots jusqu'à l'amarrage}
        \label{fig:simu_control2}
    \end{minipage}
\end{figure}


\section{Algorithme de Traitement d'Image Sonar}

Le sonar ne fournit pas directement la position de la cage, mais une image d'intensité acoustique brute. Cette section détaille la chaîne de traitement permettant d'extraire les coordonnées $(x, y, \psi)$ de la cible.

\subsection{Choix de la cage et Configuration des Réflecteurs}

\subsection{Chaîne de Traitement d'Image}

\begin{enumerate}
    \item \textbf{Pré-traitement et Filtrage :} \textit{[Instruction : Expliquez comment vous nettoyez l'image (filtre médian, seuillage, suppression du bruit de fond).]}
    \item \textbf{Extraction de Caractéristiques (Feature Extraction) :} \textit{[Instruction : Détaillez comment l'algorithme "trouve" la cage. Est-ce par détection de blobs (zones brillantes des réflecteurs), par détection de lignes (Transformée de Hough), ou par Machine Learning ?]}
\end{enumerate}

\subsection{Estimation de la Position Relative}

\textit{[Instruction : Une fois les points d'intérêt trouvés dans l'image, expliquez la formule trigonométrique ou l'algorithme (ex: PnP) qui convertit les pixels (u,v) en mètres (x,y) dans le repère du robot.]}


\section{Résultats Théoriques}

Cette section présente les performances attendues du système complet (Perception + Commande) en simulation.

\textit{[Instruction : Insérez ici une description des graphiques attendus. Par exemple : "La Figure X montre la décroissance de l'erreur de position en fonction du temps. On observe une convergence en moins de 30 secondes avec un dépassement nul." Discutez de la robustesse théorique : comment le système réagit-il si on ajoute du bruit sur les mesures sonar dans la simulation ?]}
