\renewcommand{\abstractname}{Résumé}
\begin{abstract}
Ce rapport présente le travail réalisé au sein du AASS Research Center (AMM lab) de l’Université d’Örebro 
pour améliorer l’intégration de mains robotisées sur un bras compliant et développer une chaîne perception-vers-prise 
basée sur ROS~2. L’objectif initial était d’implémenter des interfaces réutilisables pour des mains multi-doigts 
(notamment la Schunk SIH) dans l’architecture \texttt{ros2\_control}, et d’évaluer si la préemption de la fermeture 
de la main pendant le mouvement du bras (\textit{in-motion preemption}) peut surpasser la stratégie classique « atteindre puis saisir ». 
Après des défaillances matérielles successives de la Schunk SIH, le projet a pivoté vers des expérimentations 
avec la QB SoftHand puis la conception et l’assemblage d’une main à trois doigts (adaptation du Yale OpenHand Model O).

\vspace{0.5cm}

La méthodologie a consisté à développer des interfaces matérielles \texttt{ros2\_control} pour plusieurs mains, 
à décrire et publier les modèles URDF correspondants, et à mettre en place des paquets reproductibles dans un dépôt GitLab. 
Côté perception, une caméra Intel RealSense D435i a été utilisée pour le filtrage de nuages de points, la segmentation 
et l’estimation de repères d’objets par ACP (PCA). Pour l’action, un contrôleur de conformité cartésienne (compliance control) pour le bras Franka Emika Panda 
a été intégré et testé, incluant un contrôle interactif via RViz, la compensation de gravité, un pont capteurs de force/torque 
et un service de validation de prise basé sur la mesure de forces. L’orchestration globale est assurée par une machine à états finis 
coordonnant perception, planification et exécution.

\vspace{0.5cm}

Les résultats incluent des briques logicielles modulaires et documentées (interfaces, URDF, scripts de test, FSM), 
une pipeline perception-vers-grasp opérationnelle et un prototype matériel de main trois-doigts. 
Ce travail met également en lumière les limites matérielles rencontrées et ouvre des perspectives de recherche pour l'université d’Örebro, 
notamment l’évaluation de la préemption en mouvement et l’utilisation d’approches d’apprentissage automatique 
pour la perception.
\end{abstract}

\vspace{1cm}

% Abstract in English
\renewcommand{\abstractname}{Abstract}
\begin{abstract}
This report presents the work carried out at the AASS Research Center (AMM lab) of Örebro University 
to improve the integration of robotic hands on a compliant arm and to develop a perception-to-grasping pipeline 
based on ROS~2. The initial objective was to implement reusable interfaces for multi-fingered hands 
(notably the Schunk SIH) within the \texttt{ros2\_control} architecture, and to evaluate whether grasp closure 
preemption during arm motion (\textit{in-motion preemption}) could outperform the classical reach-then-grasp strategy. 
After successive hardware failures with the Schunk SIH, the project shifted towards experiments with the QB SoftHand 
and eventually the design and assembly of a three-finger robotic gripper (adaptation of the Yale OpenHand Model O).

\vspace{0.5cm}

The methodology involved developing \texttt{ros2\_control} hardware interfaces for several hands, 
describing and publishing their URDF models, and setting up reproducible GitLab packages. 
On the perception side, an Intel RealSense D435i camera was used for point cloud filtering, segmentation, 
and object frame estimation via PCA. On the action side, a Cartesian compliance controller for the Franka Emika Panda arm 
was integrated and tested, including interactive control in RViz, gravity compensation, a force-torque bridge, 
and a grasp validation service based on force measurements. The global orchestration relied on a finite state machine 
coordinating perception, planning, and execution.

\vspace{0.5cm}

The results include modular and well-documented software components (interfaces, URDF, test scripts, FSM), 
an operational perception-to-grasp pipeline, and a physical prototype of a three-finger hand. 
This work also highlights the hardware limitations encountered and suggests future perspectives, 
such as evaluation of in-motion grasp preemption, and use of machine learning for perception tasks.
\end{abstract}

\vspace{1cm}