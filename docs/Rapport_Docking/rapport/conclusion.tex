\newpage
\chapter*{Conclusion}
\addcontentsline{toc}{chapter}{Conclusion}
\section{Conclusion Générale}

Ce rapport a permis d'explorer la faisabilité technique de l'amarrage autonome d'un ROV d'inspection en utilisant uniquement un sonar d'imagerie acoustique pour la phase terminale.

\subsection{Synthèse des Réalisations}

\textit{[Instruction : Résumez ici les jalons techniques validés ("ce qui est fait"). Mentionnez que la chaîne logicielle complète (ROS2 + Driver Oculus + Algorithme de détection) est opérationnelle. Soulignez la réussite de l'intégration mécanique et électronique sur le BlueROV Heavy. Rappelez les résultats clés de la simulation et des essais en bassin qui valident l'approche théorique. Si l'amarrage a réussi partiellement ou totalement à Guerlédan, mettez-le en avant comme une preuve de concept majeure en milieu réel turbide.]}

\subsection{Limites et Perspectives}

Malgré les résultats encourageants, plusieurs verrous subsistent pour atteindre une autonomie industrielle fiable.

\textit{[Instruction : Discutez honnêtement des "choses non faites" ou des limites actuelles. Par exemple : la détection est-elle robuste si la cage est vue sous un angle critique ? La loi de commande actuelle permet-elle de lutter contre des courants marins violents ? Proposez des pistes pour la suite : fusionner le sonar avec un DVL pour stabiliser la vitesse, implémenter une commande plus avancée (type MPC), ou utiliser des techniques d'apprentissage profond (Deep Learning) pour améliorer la reconnaissance de la cage dans des environnements acoustiques complexes.]}
