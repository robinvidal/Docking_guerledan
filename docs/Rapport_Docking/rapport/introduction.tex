\chapter*{Introduction Générale}
\addcontentsline{toc}{chapter}{Introduction Générale}

\section{Contexte Stratégique et Nécessité du Docking}

L'évolution de la robotique sous-marine marque une transition fondamentale, passant de missions expéditionnaires ponctuelles à des déploiements persistants de systèmes résidents. Dans ce nouveau paradigme, les véhicules sous-marins autonomes (AUV) et les véhicules filoguidés (ROV) ne sont plus systématiquement récupérés en surface après chaque intervention. Ils doivent au contraire être capables d'effectuer une phase d'approche et de capture de manière autonome par des infrastructures immergées pour recharger leurs batteries et transférer les données collectées, condition indispensable au fonctionnement des systèmes dits \textit{long-endurance}.

Cette capacité d'amarrage autonome, ou \textit{docking}, constitue l’un des verrous technologiques majeurs pour l’essor de l’économie bleue. Plusieurs travaux récents soulignent que cette manœuvre est comparable à un appontage ou à un ravitaillement en vol, mais réalisée dans un fluide dense et perturbé \cite{yazdani2020_survey_docking}. La difficulté centrale réside dans l’alignement précis du véhicule avec l’axe d’entrée de la station, généralement avec des tolérances centimétriques, tout en compensant les effets du courant, les interactions hydrodynamiques de proximité et l'incertitude de navigation. A cela s'ajoute la difficulté de percevoir la station d'accueil et de se s'orienter par rapport à elle dans un environnement sous-marin souvent hostile.

L’environnement sous-marin impose en effet des contraintes sévères qui rendent les méthodes de guidage traditionnelles souvent inopérantes. Plusieurs études proposent alors de recourir à l’imagerie acoustique, plus robuste en conditions dégradées \cite{palomeras2016_iauv_panel, uchihori2019_3dsonar_docking}. Le sonar apparaît ainsi comme un capteur incontournable pour assurer la perception longue et moyenne portée.

\section{État de l'Art et Positionnement du Sujet}

La littérature scientifique décompose généralement la séquence de docking en trois phases, chacune reposant sur une modalité de capteur spécifique \cite{yazdani2020_survey_docking, wang2023_acoustic_mfls_docking} :

\begin{enumerate}
    \item \textbf{Phase de ralliement (longue portée)} : guidage par systèmes acoustiques (LBL/USBL), efficaces mais peu précis à courte distance.
    \item \textbf{Phase d'approche (moyenne portée, 3--100 m)} : détection de la station via sonar d'imagerie frontale (FLS), fournissant une géométrie exploitable même en eaux troubles.
    \item \textbf{Phase terminale (courte portée, < 3 m)} : la plupart des travaux basculent sur un guidage visuel ou inertiel \cite{palomeras2016_iauv_panel}, le sonar étant traditionnellement jugé difficile à exploiter en raison des zones d’ombre, des multi-trajets et de la saturation acoustique.
\end{enumerate}

Les travaux utilisant un sonar 3D pour le docking \cite{uchihori2019_3dsonar_docking} ou combinant imagerie acoustique et communication \cite{wang2023_acoustic_mfls_docking} montrent qu'une approche terminale à l'aide d'un guidage sonar peut être une stratégie prometteuse, mais rarement poussée jusqu’à l’entrée dans le dock lui-même. Dans des environnements fortement turbides, comme les ports, les estuaires ou des conduites, les caméras deviennent rapidement inopérantes, ce qui renforce la pertinence d’un guidage entièrement acoustique.

Un tel positionnement impose cependant des défis spécifiques : traitement d’image acoustique à très courte portée, extraction robuste de la pose relative, et conception mécanique du dock pour maximiser sa signature acoustique. Ces problématiques rejoignent également des approches plus théoriques visant à garantir la robustesse du docking en présence d’incertitudes, qu’elles soient liées au capteur ou à la dynamique \cite{bourgois2021_safe_underwater_docking}.

Les performances théoriques d’un système d’amarrage guidé au sonar peuvent être déduites de l’état de l’art. Yazdani \cite{yazdani2020_survey_docking} et Palomeras \cite{palomeras2016_iauv_panel} montrent que l’information la plus déterminante pour un docking acoustique est l’orientation relative de la station, l’estimation angulaire étant généralement plus sensible au bruit que la distance.

Wang et al. \cite{wang2023_acoustic_mfls_docking} démontrent qu’une approche géométrique simple, basée sur la détection de points brillants dans l’image sonar, permet une convergence fiable lorsque la signature acoustique de la cage est bien définie. Une commande locale de type proportionnelle ou PD suffit alors pour stabiliser la phase d’approche lente.

Enfin, les travaux d’Uchihori \cite{uchihori2019_3dsonar_docking} confirment que l’utilisation de deux réflecteurs passifs produit une signature robuste permettant une estimation stable de l’orientation et une précision d’amarrage compatible avec les exigences opérationnelles.

Ces résultats convergent vers un constat : un amarrage entièrement acoustique est techniquement réalisable, à condition que la station soit conçue pour offrir une signature exploitable et que l'environnement demeure relativement calme. 

\textbf{Le présent travail s'inscrit dans cette perspective en proposant un système de docking intégralement guidé par sonar, y compris durant la phase terminale.} Il s'appuie sur un BlueROV2 Heavy (8 propulseurs) équipé d'un sonar Oculus M750d, piloté sous ROS~2, avec pour objectif d'accomplir un docking entièrement autonome dans une cage en aluminium munie de réflecteurs passifs.


\section{Plan}

Ce rapport détaille la méthodologie et les résultats de cette démarche à travers deux parties principales.

\paragraph{Partie I : Étude Théorique et Algorithmique}
Cette première partie posera les fondements du système développé :
\begin{itemize}
    \item Définition des hypothèses de travail : modèle du véhicule, caractéristiques du sonar, contraintes environnementales.
    \item Proposition de la loi de commande permettant le suivi de trajectoire et la stabilisation malgré les perturbations.
    \item Validation préliminaire via simulation, afin de tester et affiner les stratégies de guidage.
    \item Développement des méthodes de traitement d'image sonar pour extraire la pose relative de la cage.
    \item Présentation des résultats théoriques et analyse des performances attendues.
\end{itemize}

\paragraph{Partie II : Matériel et Expérimentation}
Cette seconde partie se concentrera sur la mise en œuvre pratique :
\begin{itemize}
    \item Présentation de l’architecture matérielle : BlueROV modifié et intégration du sonar Oculus M750d.
    \item Description des deux campagnes d'essais : expérimentation en bassin puis tests en lac.
    \item Analyse des résultats expérimentaux en confrontant performances réelles et prédictions issues de la partie théorique.
\end{itemize}

Le rapport s’achèvera par une conclusion générale synthétisant les contributions apportées et les pistes d’amélioration pour fiabiliser totalement l’amarrage autonome par sonar.
